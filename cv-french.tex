\documentclass[a4paper, 10pt]{article}

% Fix margins and encoding
\usepackage[hmargin=1.25cm, vmargin=1.25cm]{geometry}
\usepackage[utf8]{inputenc}

% Import other packages
\usepackage{xunicode}
\usepackage{xcolor}
\usepackage{fontspec}
\usepackage{graphicx}
\usepackage{tikz}
\usepackage{url}
\usepackage{enumitem}

% Define colors
\definecolor{lightg}{HTML}{999999}
\definecolor{medg}{HTML}{666666}
\definecolor{darkg}{HTML}{333333}
\definecolor{hyperlink}{rgb}{0,0.2,0.6}
\definecolor{maroon}{rgb}{0.5,0,0}

\definecolor{noteone}{HTML}{999999}
\definecolor{notetwo}{HTML}{848484}
\definecolor{notethree}{HTML}{424242}
\definecolor{notefour}{HTML}{212121}
\definecolor{notefive}{HTML}{000000}

% Custom bullet
\renewcommand{\labelitemi}{\textcolor{lightg}{\symbol{"2022}}}

% Commands to display nodes on a scale of five
\newcommand{\fivenotes}{%
\textcolor{noteone}{\symbol{"2022}}
\textcolor{notetwo}{\symbol{"2022}}
\textcolor{notethree}{\symbol{"2022}}
\textcolor{notefour}{\symbol{"2022}}
\textcolor{notefive}{\symbol{"2022}}
}

\newcommand{\fournotes}{%
\textcolor{noteone}{\symbol{"2022}}
\textcolor{notetwo}{\symbol{"2022}}
\textcolor{notethree}{\symbol{"2022}}
\textcolor{notefour}{\symbol{"2022}}
\textcolor{white}{\symbol{"2022}}
}

\newcommand{\threenotes}{%
\textcolor{noteone}{\symbol{"2022}}
\textcolor{notetwo}{\symbol{"2022}}
\textcolor{notethree}{\symbol{"2022}}
\textcolor{white}{\symbol{"2022}}
\textcolor{white}{\symbol{"2022}}
}

\newcommand{\twonotes}{%
\textcolor{noteone}{\symbol{"2022}}
\textcolor{notetwo}{\symbol{"2022}}
\textcolor{white}{\symbol{"2022}}
\textcolor{white}{\symbol{"2022}}
\textcolor{white}{\symbol{"2022}}
}

\newcommand{\onenote}{%
\textcolor{noteone}{\symbol{"2022}}
\textcolor{white}{\symbol{"2022}}
\textcolor{white}{\symbol{"2022}}
\textcolor{white}{\symbol{"2022}}
\textcolor{white}{\symbol{"2022}}
}

% Set fonts
\defaultfontfeatures{Mapping=tex-text}
\setromanfont[Mapping=tex-text]{Hoefler Text}
\newcommand{\myname}[1]{\fontspec{Helvetica Neue}\fontsize{32pt}{0pt}\selectfont #1}%
\newcommand{\mytitle}[1]{\fontspec{Helvetica Neue UltraLight}\fontsize{32pt}{0pt}\selectfont #1}%
\newcommand{\mysection}[1]{\fontspec{Hoefler Text}\fontsize{24pt}{0pt}\selectfont #1}%
\newcommand{\myurl}[1]{\fontspec{Gill Sans}\fontsize{10pt}{0pt}\selectfont #1}%
\newcommand{\mygreek}[1]{\fontspec{Symbol}\fontsize{220pt}{0pt}\selectfont #1}%
\newcommand{\overlay}[1]{\fontspec[Alternate=1, Ligatures={Common, Rare}]{Hoefler Text}\fontsize{160pt}{0pt}\selectfont #1}%

% Custom url style
\urlstyle{same}

% Display typographic grid
% \usepackage[columns=5]{cvtypogrid}

% Remove page numbering
\pagestyle{empty}

% Let's do this
\begin{document}

% Remove paragraph indentation
\parindent 0cm

% Header
  \begin{tikzpicture}[overlay, opacity=0.1]
    \draw (0.25, -0.25) node {\mygreek λ};
  \end{tikzpicture}

  % Name block
  \begin{minipage}[t]{0.59\linewidth}
    {\myname Robert Anisko,}\\
    {\parindent 2em \par \mytitle d\'eveloppeur senior}
  \end{minipage}
  \hfill
  % School block
  \begin{minipage}[t]{0.38\linewidth}
    {\large \textcolor{lightg}{ing\'enieur en informatique\\dipl\^om\'e de l'}\textcolor{darkg}{EPITA}\\\\
    \textcolor{lightg}{n\'e le} \textcolor{darkg}{6 d\'ecembre 1980}}
  \end{minipage}

~\\
~\\

% Main sections, enclosed in an unbalanced two column layout
  % Left block: experience, contact
  \begin{minipage}[t]{0.59\linewidth}
    {\textcolor{maroon}{\mysection exp\'erience professionnelle}}\\

    \par ~ \hfill \textcolor{medg}{\textit{en poste depuis fin 2004}}\\[5pt]
    {\large D\'eveloppeur senior chez \textsc{Murex}, Paris\\
    \emph{\textcolor{darkg}{D\'epartement Recherche \& D\'eveloppement}}}\\[5pt]
    D\'eveloppement d'une nouvelle plateforme logicielle: parall\'elisation et distribution des calculs; interop\'erabilit\'e transparente entre langages Java, C++ et C\#; programmation r\'eactive par d\'ependances; aggr\'egation de donn\'ees provenant de sources externes; persistance. Conception d'un prototype de visualisation interactive de la propagation des calculs pour cette plateforme.\\[5pt]
    Pr\'ec\'edemment membre de l'\'equipe market data, en charge d'unifier les biblioth\`eques d'acc\`es aux param\`etres de march\'e dans \textsc{Mx}, solution pour le trading et la gestion de risque.\\[5pt]
    {\textcolor{hyperlink}{\myurl | \url{http://www.murex.com}}}\\

    \par ~ \hfill \textcolor{medg}{\textit{premier semestre 2003}}\\[5pt]
    {\large Stagiaire \`a l'\textsc{Universit\'e d'Utrecht}, Pays-Bas\\
    \emph{\textcolor{darkg}{Section Software Technology}}}\\[5pt]
    Travail sur la transformation et la g\'en\'eration de programmes, conception d'un langage \'etag\'e. Construction dans un environnement d\'edi\'e des outils associ\'es \`a ce langage: interpr\'eteur, compilateur vers C++.\\[5pt]
    Prototype pr\'esent\'e au Fourth Stratego Users Day (SUD'03).\\

    \par ~ \hfill \textcolor{medg}{\textit{entre 2001 et 2004}}\\[5pt]
    {\large Dans le cadre du \textsc{Lrde}, Paris}\\[5pt]
    Participation au d\'eveloppement en C++ d'un compilateur pour le langage jouet Tiger, utilis\'e comme support pour l'enseignement des langages formels et de la programmation orient\'ee objet. \'Ecriture en Haskell d'une machine virtuelle pour la repr\'esentation interm\'ediaire du compilateur Tiger. Contribution au g\'en\'erateur de parsers libre GNU Bison.\\[5pt]
    D\'eveloppement initial d'un syst\`eme de transformation de programmes pour le language C++, incluant la sp\'ecification d'une grammaire et d'une collection de filtres d'aide \`a l'analyse syntaxique.\\[5pt]
    Travail pr\'esent\'e aux Fifth Stratego User Days (SUD'04).\\

    {\large Charg\'e de cours \`a l'\textsc{Epita}, Paris}\\[5pt]
    Cours d'algorithmique et s\'eminaire intensif d'Objective Caml \`a des \'etudiants de cycle pr\'eparatoire, charg\'e de travaux pratiques en Pascal.

    \vspace{6em}

    {\large 
	\begin{tabular}{@{}rl}
    \textcolor{maroon}{adresse} &
    11 rue Cels, 75014 Paris\\
    \textcolor{maroon}{t\'el\'ephone} &
    06 72 19 77 67\\
    \textcolor{maroon}{e-mail} &
    {\myurl \url{robert.anisko@gmail.com}}\\
    \textcolor{maroon}{web} &
    {\myurl \url{http://www.robertanisko.com}}\\
    \end{tabular}}
  \end{minipage}
  \hfill
  % Right block: education, skills, others
  \begin{minipage}[t]{0.38\linewidth}
    {\textcolor{maroon}{\mysection formation}}\\

    \par ~ \hfill \textcolor{medg}{\textit{de 2001 \`a 2003}}\\[5pt]
    Membre du \textbf{LRDE}, Laboratoire de Recherche et D\'eveloppement de l'EPITA.\\[5pt]
    {\textcolor{hyperlink}{\myurl | \url{http://www.lrde.epita.fr}}}\\

    \par ~ \hfill \textcolor{medg}{\textit{de 1998 \`a 2003}}\\[5pt]
    Dipl\^om\'e de l'\'Ecole Pour l'Informatique et les Techniques Avanc\'ees, \textbf{EPITA}, au sein de la premi\`ere promotion de la sp\'ecialisation calcul scientifique et image.\\[5pt]
    {\textcolor{hyperlink}{\myurl | \url{http://www.epita.fr}}}\\[12.5pt]
    %
    % \par ~ \hfill \textcolor{medg}{\textit{1998}}\\[5pt]
	% Baccalaur\'eat S au lyc\'ee Louise Michel de Champigny-sur-Marne.\\[12.5pt]
    {\textcolor{maroon}{\mysection comp\'etences}}\\
    
    \begin{itemize}[leftmargin=10.5pt, rightmargin=0pt, labelsep=5pt, itemsep=0pt, topsep=0pt]
      \item C/C++ avec STL et Boost \hfill \fivenotes
      \item Haskell \hfill \threenotes
      \item Java \hfill \fournotes
      \item C\# \hfill \threenotes
      \item OpenGL \hfill \threenotes
      \item DirectX et XNA \hfill \fournotes
      \item \LaTeX \hfill \threenotes
      \item HTML/CSS/JavaScript \hfill \threenotes
	\end{itemize}

	~\\[4pt]
    {\textcolor{maroon}{\mysection centres d'int\'er\^et}}\\[-5pt]

    \begin{itemize}[leftmargin=10.5pt, rightmargin=0pt, labelsep=5pt, itemsep=0pt, topsep=0pt]
      \item Programmation fonctionnelle et langages modernes: Scala, Ruby.
      \item Programmation graphique, visualisation de donn\'ees, design, typographie.
      \item Programmation parall\`ele/distribu\'ee/acteurs.
      \item D\'eveloppement web: Node.js, jQuery, Redis.
    \end{itemize}

	~\\[4pt]
    {\textcolor{maroon}{\mysection divers}}\\[-5pt]

    \begin{itemize}[leftmargin=10.5pt, rightmargin=0pt, labelsep=5pt, itemsep=0pt, topsep=0pt]
      \item Anglais courant, pratique quotidienne.
      \item Guitariste, a fait partie d'un groupe amateur pendant plusieurs ann\'ees.
      \item Pratique le hockey et le krav-maga.
    \end{itemize}
  \end{minipage}
  
  % Footer with contacts
  \begin{tikzpicture}[overlay, opacity=0.1]
    \draw (7, -0.015) node {\overlay contact};
  \end{tikzpicture}
\end{document}
